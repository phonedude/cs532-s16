\section{Problem 2}

\subsection{Question}
Create a blog-term matrix.  Start by grabbing 100 blogs; include:\\
\\
\url{http://f-measure.blogspot.com/}\\
\url{http://ws-dl.blogspot.com/}\\
\\
and grab 98 more as per the method shown in class.  Note that this method randomly chooses blogs and each student will separately do this process, so it is unlikely that these 98 blogs will be shared among students.  In other words, no sharing of blog data.  Upload to github your code for grabbing the blogs and provide a list of blog URIs, both in the report and in github..\\
\\
Use the blog title as the identifier for each blog (and row of the matrix).  Use the terms from every item/title (RSS) or entry/title (Atom) for the columns of the matrix.  The values are the frequency of occurrence.  Essentially you are replicating the format of the \enquote{blogdata.txt} file included with the PCI book code.  Limit the number of terms to the most \enquote{popular} (i.e., frequent) 500 terms, this is *after* the criteria on p. 32 (slide 7) has been satisfied.

\subsection{Answer}
The ascii and jpeg dendrograms were created using the code shown in Listing \ref{listing:clust:main}, which is modeled after the example from class. 

\lstinputlisting[language=Python, caption={creating the dendrograms}, label=listing:clust:main,linerange={286-293},firstnumber=286]{clusters.py}

The {\tt readfile} function shown in Listing \ref{listing:clust:read} was used to read the data that was compiled from Question 1 into memory where it is then processed by the {\tt hcluster} function found in Listing \ref{listing:clust:hclust} to produce the clustered representation of the blogs.

\lstinputlisting[language=Python, caption={creating the dendrograms}, label=listing:clust:read,linerange={3-16},firstnumber=3]{clusters.py}

\lstinputlisting[language=Python, caption={hcluster function}, label=listing:clust:hclust,linerange={48-88},firstnumber=48]{clusters.py}

The {\tt printclust} function from Listing \ref{listing:clust:printclust} prints the ascii dendrogram of the cluster object parameter to sys.stdout, which is redirected to write to a file with the code in Listing \ref{listing:clust:main}.

\lstinputlisting[language=Python, caption={printclust function}, label=listing:clust:printclust,linerange={90-103},firstnumber=90]{clusters.py}

The {\tt drawdendrogram} function from Listing \ref{listing:clust:drawdendro} creates a jpeg image of the cluster, which is shown in Figure \ref{fig:clust}.

\lstinputlisting[language=Python, caption={drawdendrogram function}, label=listing:clust:drawdendro,linerange={122-139},firstnumber=122]{clusters.py}

\begin{figure}[h!]
\centering
\fbox{\includegraphics[scale=0.265]{q2/blogclust.jpg}}
\caption{blog dendrogram}
\label{fig:clust}
\end{figure}