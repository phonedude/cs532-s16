\section{Problem 3}

\subsection{Question}
Create a blog-term matrix.  Start by grabbing 100 blogs; include:\\
\\
\url{http://f-measure.blogspot.com/}\\
\url{http://ws-dl.blogspot.com/}\\
\\
and grab 98 more as per the method shown in class.  Note that this method randomly chooses blogs and each student will separately do this process, so it is unlikely that these 98 blogs will be shared among students.  In other words, no sharing of blog data.  Upload to github your code for grabbing the blogs and provide a list of blog URIs, both in the report and in github..\\
\\
Use the blog title as the identifier for each blog (and row of the matrix).  Use the terms from every item/title (RSS) or entry/title (Atom) for the columns of the matrix.  The values are the frequency of occurrence.  Essentially you are replicating the format of the \enquote{blogdata.txt} file included with the PCI book code.  Limit the number of terms to the most \enquote{popular} (i.e., frequent) 500 terms, this is *after* the criteria on p. 32 (slide 7) has been satisfied.

\subsection{Answer}

Using the code in Listing \ref{listing:kclust} kclustering was performed with values for {\it n = 5}, {\it n = 10} and {\it n = 20}. The main function calls the kcluster function, which is shown in Listing \ref{listing:kclustdef}.

\lstinputlisting[language=Python, caption={kclustering main}, label=listing:kclust, linerange={295-305},firstnumber=295]{clusters.py}

\lstinputlisting[language=Python, caption={kcluster function}, label=listing:kclustdef, linerange={174-212},firstnumber=174]{clusters.py}

The output is shown in Listing \ref{listing:kclust:out}. As the output reads, a kcluster with {\it n = 5} required eight iterations, {\it n = 10} required five iterations and {\it n = 20} also required five iterations.

\lstinputlisting[language=bash, caption={output of kclustering algorithm}, label=listing:kclust:out]{q3/kclust.txt}