\section{Problem 3}

\subsection{Question}
Re-run question 2, but this time with proper TFIDF calculations instead of the hack discussed on slide 7 (p. 32).  Use the same 500 words, but this time replace their frequency count with TFIDF scores as computed in assignment \#3. Document the code, techniques, methods, etc. used to generate these TFIDF values.  Upload the new data file to github.\\
\\
Compare and contrast the resulting dendrogram with the dendrogram from question \#2.\\
\\
Note: ideally you would not reuse the same 500 terms and instead come up with TFIDF scores for all the terms and then choose the top 500 from that list, but I'm trying to limit the amount of work necessary.

\subsection{Answer}

Using the code in Listing \ref{listing:kclust} kclustering was performed with values for {\it n = 5}, {\it n = 10} and {\it n = 20}. The main function calls the kcluster function, which is shown in Listing \ref{listing:kclustdef}.

\lstinputlisting[language=Python, caption={kclustering main}, label=listing:kclust, linerange={295-305},firstnumber=295]{clusters.py}

\lstinputlisting[language=Python, caption={kcluster function}, label=listing:kclustdef, linerange={174-212},firstnumber=174]{clusters.py}

The output is shown in Listing \ref{listing:kclust:out}. As the output reads, a kcluster with {\it n = 5} required eight iterations, {\it n = 10} required five iterations and {\it n = 20} also required five iterations.

\lstinputlisting[language=bash, caption={output of kclustering algorithm}, label=listing:kclust:out]{q3/kclust.txt}