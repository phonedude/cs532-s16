\section{Problem 4}

\subsection{Question}
Create a blog-term matrix.  Start by grabbing 100 blogs; include:\\
\\
\url{http://f-measure.blogspot.com/}\\
\url{http://ws-dl.blogspot.com/}\\
\\
and grab 98 more as per the method shown in class.  Note that this method randomly chooses blogs and each student will separately do this process, so it is unlikely that these 98 blogs will be shared among students.  In other words, no sharing of blog data.  Upload to github your code for grabbing the blogs and provide a list of blog URIs, both in the report and in github..\\
\\
Use the blog title as the identifier for each blog (and row of the matrix).  Use the terms from every item/title (RSS) or entry/title (Atom) for the columns of the matrix.  The values are the frequency of occurrence.  Essentially you are replicating the format of the \enquote{blogdata.txt} file included with the PCI book code.  Limit the number of terms to the most \enquote{popular} (i.e., frequent) 500 terms, this is *after* the criteria on p. 32 (slide 7) has been satisfied.

\subsection{Answer}

With the code in Listing \ref{listing:scaledownmain}, multidimensional scaling (MDS) was used to create a two-dimensional visualization of the blog distance graph. This code calls the {\tt scaledown} function, which is shown in Listing \ref{listing:scaledownfunc}. The algorithm continues until the error factor stops decreasing, as shown in the output in Listing \ref{scaledown}. 

\lstinputlisting[language=Python, caption={main for scaledown}, label=listing:scaledownmain,linerange={306-307},
firstnumber=306]{clusters.py}

\lstinputlisting[language=Python, caption={scaledown function}, label=listing:scaledownfunc,linerange={224-272},firstnumber=224]{clusters.py}

The {\tt scaledown} function returns the coordinates for each of the blogs in 2D space. This data was then used with the {\tt draw2d} function in Listing \ref{listing:draw2d}, which produced the two-dimensional visualation created from the MDS algorithm, as shown in Figure \ref{fig:blogs2d}.

\lstinputlisting[language=Python, caption={draw2d function}, label=listing:draw2d,linerange={274-281},firstnumber=274]{clusters.py}

\begin{figure}[h!]
\centering
\fbox{\includegraphics[trim=0 0 0 500, clip, scale=0.2]{q4/blogs2d.jpg}}
\caption{MDS 2d visualization}
\label{fig:blogs2d}
\end{figure}
\clearpage
\lstinputlisting[language=Bash, caption={scaledown output}, label=scaledown]{q4/scaledown.txt}