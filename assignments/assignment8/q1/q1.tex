\section{Question 1}

\subsection{Question}
Create a blog-term matrix.  Start by grabbing 100 blogs; include:\\
\\
\url{http://f-measure.blogspot.com/}\\
\url{http://ws-dl.blogspot.com/}\\
\\
and grab 98 more as per the method shown in class.  Note that this method randomly chooses blogs and each student will separately do this process, so it is unlikely that these 98 blogs will be shared among students.  In other words, no sharing of blog data.  Upload to github your code for grabbing the blogs and provide a list of blog URIs, both in the report and in github..\\
\\
Use the blog title as the identifier for each blog (and row of the matrix).  Use the terms from every item/title (RSS) or entry/title (Atom) for the columns of the matrix.  The values are the frequency of occurrence.  Essentially you are replicating the format of the \enquote{blogdata.txt} file included with the PCI book code.  Limit the number of terms to the most \enquote{popular} (i.e., frequent) 500 terms, this is *after* the criteria on p. 32 (slide 7) has been satisfied.

\subsection{Answer}
To complete this assignment, a blog word count matrix was required. To start off, a list of blog URIs was obtained using the method described in class, implemented as the {\tt get\_uris.py} script, which can be found in Appendix A, Listing \ref{listing:get_uris}. Two default blogs, F-Measure and the Old Dominion Web Science and Digital Libraries blogs, were added as defaults to the initial URI list and then, using the seed URI provided (Listing \ref{listing:default}), the remaining 98 URIs from random blogs within the blogger.com family were added.
\vspace{2mm}
\lstinputlisting[language=Python, caption={referenced variables in get\_uris.py}, label=listing:default, linerange={7-8}, firstnumber=7]{q1/get_uris.py}

The {\tt get\_uris main} function in Listing \ref{listing:geturis:main} was the driver that called the {\tt get\_atom} function (shown in Listing \ref{listing:geturis:atom}) to extract the atom \cite{atom} URIs from each blog and add them to the set of URIs with the {\tt add\_uri} function, shown in Listing \ref{listing:geturis:adduri}.
\lstinputlisting[caption={Sample list of Blog URI's}, label=listing:bloguris, linerange={19-28}, firstnumber=19]{q1/blog_uris.txt}

\clearpage

\lstinputlisting[language=Python, caption={main for get\_uris.py}, label=listing:geturis:main, linerange={27-39}, firstnumber=27]{q1/get_uris.py}


\lstinputlisting[language=Python, caption={get\_atom function}, label=listing:geturis:atom, linerange={10-19}, firstnumber=10]{q1/get_uris.py}

\lstinputlisting[language=Python, caption={add\_uri function}, label=listing:geturis:adduri, linerange={21-25}, firstnumber=21]{q1/get_uris.py}

After the full list of 100 URIs was obtained, page counts for each blog were extracted and saved to a file called {\tt pagecounts} using the {\tt matrix.py} script. This script is a modified version of {\tt generatefeedvectors.py} from the book {\it Programming Collective Intelligence} \cite{pci} and can be found in full in Appendix A, Listing \ref{listing:matrix}. 

The code responsible for downloading the blogs and counting the words in each is shown in Listing \ref{listing:matrix:get}, which calls the {\tt get\_titles}, {\tt get\_words} and {\tt get\_next} functions found in Listing \ref{listing:matrix:gettitles}. This code loops over the list of URIs that was obtained with the {\tt get\_uris.py} script (Listing \ref{listing:get_uris}), parses each entry, and extracts all the words in each entry's title. These word counts are then saved as a python dictionary to the hard drive for later use. 

\lstinputlisting[language=Python, caption={looping over the URIs}, label=listing:matrix:get, linerange={95-105}, firstnumber=95]{matrix.py}

\clearpage

\lstinputlisting[language=Python, caption={processing each blog}, label=listing:matrix:gettitles, linerange={9-38}, firstnumber=9]{matrix.py}

The parsed results were then read by the code in Listing \ref{listing:matrix:loadwrite}. This code used the {\tt load\_data} and {\tt build\_wordlist} functions in Listing \ref{listing:matrix:loaddata} and \ref{listing:matrix:buildwordlist} to read each of the blog word counts and then created four collections to organize them all:

\begin{enumerate}
	\item {\tt apcount}: A dictionary containing the count for all words combined
	\item {\tt wordcounts}: A dictionary containing each blog's individual word count
	\item {\tt pagecounts}: A dictionary containing each blog's page count
	\item {\tt wordlist}: A list containing all of the words found in each blog
\end{enumerate}

\lstinputlisting[language=Python, caption={creating the blog data matrix}, label=listing:matrix:loadwrite, linerange={107-115}, firstnumber=107]{matrix.py}

\clearpage

\lstinputlisting[language=Python, caption={loading the data}, label=listing:matrix:loaddata, linerange={50-69}, firstnumber=50]{matrix.py}

\lstinputlisting[language=Python, caption={building the master wordlist}, label=listing:matrix:buildwordlist, linerange={71-77}, firstnumber=71]{matrix.py}

The code in Listing \ref{listing:matrix:writedata} then created the matrix using the {\tt write\_data} function using the data structures that store the blog word counts.

\lstinputlisting[language=Python, caption={writing the data}, label=listing:matrix:writedata, linerange={79-93}, firstnumber=79]{matrix.py}