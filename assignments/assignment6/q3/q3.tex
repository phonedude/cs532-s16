\section{Problem 3}

\subsection{Question}
\vspace*{10pt}
\justify 
Use D3 to visualize your Twitter followers.  Use my twitter account
(\enquote{{@}phonedude\_mln}) if you do not have $>=$ 50 followers.  For example,
{@}hvdsomp follows me, as does {@}mart1nkle1n.  They also follow each
other, so they would both have links to me and links to each other.\\
\\
To see if two users follow each other, see:\\
\url{https://dev.twitter.com/rest/reference/get/friendships/show}\\
\\
Attractiveness of the graph counts!  Nodes should be labeled (avatar
images are even better), and edge types (follows, following) should
be marked.\\
\\
Note: for getting GitHub to serve HTML (and other media types), see:\\
\url{http://stackoverflow.com/questions/6551446/can-i-run-html-}\\
\url{files-directly-from-github-instead-of-just-viewing-their-source}\\
\\
Be sure to include the URI(s) for your D3 graph in your report. 


\subsection{Answer}
Using an example from the primary author of the D3 JavaScript library, Mike Bostok \cite{d3:bostok12}, a graph was created of Zachary's Karate Club using the pickled \cite{py:pickle} dataset found at \url{http://nexus.igraph.org/api/dataset_info?id=1&format=html}. The D3 library provides a force-directed graphing layout \cite{d3:force14}, which was used to display the graph. A transition from the initial graph, shown in Figure \ref{fig:init_graph}, to the graph after the split of the karate club, shown in Figure \ref{fig:split_graph}, was created using standard JavaScript.

\begin{figure}[h!]
\centering
\fbox{\includegraphics[scale=0.4]{q3/init_graph.pdf}}
\caption{Initial Graph}
\label{fig:init_graph}
\end{figure}

\begin{figure}[h!]
\centering
\fbox{\includegraphics[scale=0.4]{q3/split_graph.pdf}}
\caption{Graph After Split}
\label{fig:split_graph}
\end{figure}

\clearpage

The dataset was first parsed into matrices using the {\tt build\_matrix} function, shown in Listing \ref{listing:convert}. These matrices were converted into python dictionary objects, which are pickled \cite{py:pickle} into the json format \cite{json}. This output was used as the input for the JavaScript code, which uses the D3 library \cite{d3} to create the graphs.

The python code to produce the json data is shown in Listing \ref{listing:convert}.

\lstinputlisting[language=Python, caption={Data Converter}, label=listing:convert]{q3/convert.py}

\clearpage

The JavaScript code to produce the graph is shown in Listing \ref{listing:build}.

\lstinputlisting[language=JavaScript, caption={Building the Graph}, label=listing:build,linerange={27-97},firstnumber=27]{q3/build_graph.html}
