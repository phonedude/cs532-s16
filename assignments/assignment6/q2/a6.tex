Take the Twitter graph you generated in question \#1 and test for
male-female homophily.  For the purposes of this question you can
consider the graph as undirected (i.e., no distinction between
\enquote{follows} and \enquote{following}).  Use the twitter name (not \enquote{screen
name}; for example \enquote{Michael L. Nelson} and not \enquote{{@}phonedude\_mln})
and programatically determine if the user is male or female.  Some
sites that might be useful:\\
\\
\url{https://genderize.io/}\\
\url{https://pypi.python.org/pypi/gender-detector/0.0.4}\\
\\
Create a table of Twitter users and their likely gender.  List any 
accounts that can't be determined and remove them from the graph.\\
\\
Perform the homophily test as described in slides 11-15, Week 7.\\
\\
Does your Twitter graph exhibit gender homophily?