\documentclass[letterpaper,10pt]{article}
\usepackage[pdftex]{graphicx}
\usepackage{listings}
\usepackage{alltt}
\usepackage{color}
\usepackage{amsmath}
\usepackage{hyperref}
\usepackage{pgfplotstable}
\usepackage{float}


\definecolor{dkgreen}{rgb}{0,0.6,0}
\definecolor{gray}{rgb}{0.5,0.5,0.5}
\definecolor{mauve}{rgb}{0.58,0,0.82}

\hypersetup{
    colorlinks,
    citecolor=black,
    filecolor=black,
    linkcolor=black,
    urlcolor=black
}
\definecolor{lightgray}{rgb}{.9,.9,.9}
\definecolor{darkgray}{rgb}{.4,.4,.4}
\definecolor{purple}{rgb}{0.65, 0.12, 0.82}

\lstdefinelanguage{JavaScript}{
  keywords={typeof, new, true, false, catch, function, return, null, catch, switch, var, if, in, while, do, else, case, break},
  keywordstyle=\color{blue}\bfseries,
  ndkeywords={class, export, boolean, throw, implements, import, this},
  ndkeywordstyle=\color{darkgray}\bfseries,
  identifierstyle=\color{black},
  sensitive=false,
  stringstyle=\color{red}\ttfamily,
  morestring=[b]',
  morestring=[b]"
}

\lstset{
   language=JavaScript,
   backgroundcolor=\color{lightgray},
   extendedchars=true,
   basicstyle=\footnotesize\ttfamily,
   showstringspaces=false,
   showspaces=false,
   numbers=left,
   numberstyle=\footnotesize,
   numbersep=9pt,
   tabsize=2,
   breaklines=true,
   showtabs=false,
   captionpos=b
}

\lstset{
	basicstyle=\footnotesize,
	breaklines=true,
	 backgroundcolor=\color{white},   % choose the background color; you must add \usepackage{color} or \usepackage{xcolor}
  basicstyle=\footnotesize,        % the size of the fonts that are used for the code
  breakatwhitespace=false,         % sets if automatic breaks should only happen at whitespace
  breaklines=true,                 % sets automatic line breaking
  captionpos=b,                    % sets the caption-position to bottom
  commentstyle=\color{dkgreen},    % comment style
  deletekeywords={...},            % if you want to delete keywords from the given language
  escapeinside={\%*}{*)},          % if you want to add LaTeX within your code
  extendedchars=true,              % lets you use non-ASCII characters; for 8-bits encodings only, does not work with UTF-8
  frame=single,	                   % adds a frame around the code
  keepspaces=true,                 % keeps spaces in text, useful for keeping indentation of code (possibly needs columns=flexible)
  keywordstyle=\color{blue},       % keyword style
  otherkeywords={*,grep,sort,head,mv,perl,chmod,...},           % if you want to add more keywords to the set
  numbers=left,                    % where to put the line-numbers; possible values are (none, left, right)
  numbersep=5pt,                   % how far the line-numbers are from the code
  numberstyle=\tiny\color{gray}, % the style that is used for the line-numbers
  rulecolor=\color{black},         % if not set, the frame-color may be changed on line-breaks within not-black text (e.g. comments (green here))
  showspaces=false,                % show spaces everywhere adding particular underscores; it overrides 'showstringspaces'
  showstringspaces=false,          % underline spaces within strings only
  showtabs=false,                  % show tabs within strings adding particular underscores
  stepnumber=1,                    % the step between two line-numbers. If it's 1, each line will be numbered
  stringstyle=\color{mauve},     % string literal style
  tabsize=2,	                   % sets default tabsize to 2 spaces
  title=\lstname                   % show the filename of files included with \lstinputlisting; also try caption instead of title
}


\begin{document} 

\begin{titlepage}

\begin{center}

\Huge{Assignment 7}

\Large{CS532-s16:  Web Sciences}

\Large{Spring 2016}

\Large{John Berlin}

\Large Generated on \today

\end{center}

\end{titlepage}
\newpage
\section*{1}
\subsection*{Question}
\begin{verbatim}

1.  Find 3 users who are closest to you in terms of age, 
gender, and occupation.  For each of those 3 users:

- what are their top 3 favorite films?
- bottom 3 least favorite films?

Based on the movie values in those 6 tables (3 users X (favorite +
least)), choose a user that you feel is most like you.  Feel 
free to note any outliers (e.g., "I mostly identify with user 123,
except I did not like ``Ghost'' at all").  

This user is the "substitute you".  
\end{verbatim}
\subsection*{Answer}
The code for this is found in listing \ref{lst:cf}
The top three users like me are:
\begin{verbatim}
user(551, 25, M, programmer)
top three movies: 
(Young Guns (1988),5),
(Star Trek: First Contact (1996),5),
(Star Trek VI: The Undiscovered Country (1991),5)
bottom three movies: 
(To Die For (1995),1),
(Naked Gun 33 1/3: The Final Insult (1994),1),
(Bram Stoker's Dracula (1992),1)

user(595, 25, M, programmer)
top three movies: 
(Willy Wonka and the Chocolate Factory (1971),5),
(Godfather, The (1972),5),
(Star Wars (1977),5)
bottom three movies: 
(Phantom, The (1996),1),
(Down Periscope (1996),1),
(Volcano (1997),1)

user(622, 25, M, programmer) -Me
top three movies: 
(Fargo (1996),5),
(Star Trek: First Contact (1996),5),
(Much Ado About Nothing (1993),5)
bottom three movies: 
(Tin Cup (1996),1),
(Under Siege 2: Dark Territory (1995),1),
(Striptease (1996),1)
\end{verbatim}
I choose user 622 because he had Fargo(ok Wade) and a Star Trek movie in his top three.
Even though I Much Ado About Nothing is not a movie id watch all the time it was very good. 
His bottom three movies I had not seen and only heard of Under Siege 2,which I thought looked dumb.
Thus I decided he was the closes fit to me.

\section*{2}
\subsection*{Question}
\begin{verbatim}
2.  Which 5 users are most correlated to the substitute you? Which
5 users are least correlated (i.e., negative correlation)?
\end{verbatim}
\subsection*{Answer}
\begin{verbatim}
Top 5 correlated users:
user(819, 59, M, administrator) correlation: 1.0
user(813, 14, F, student) correlation: 1.0
user(3, 23, M, writer) correlation: 1.0
user(762, 32, M, administrator) correlation: 0.891042111213631
user(909, 50, F, educator) correlation: 0.8703882797784892


Bottom 5 correlated users:
user(832, 24, M, technician) correlation: -1.000000000000004
user(242, 33, M, educator) correlation: -0.8017837257372769
user(31, 24, M, artist) correlation: -0.7660323462854247
user(462, 19, F, student) correlation: -0.745355992499929
user(565, 40, M, student) correlation: -0.7372097807744856
\end{verbatim}

As seen in the 5 users who are most correlated to me I tend to like more older movies which in real life is true. Thus I am more correlated to older people except the 14yr old girl. Interesting.


\section*{3}
\subsection*{Question}
\begin{verbatim}
3.  Compute ratings for all the films that the substitute you
have not seen.  Provide a list of the top 5 recommendations for films
that the substitute you should see.  Provide a list of the bottom
5 recommendations (i.e., films the substitute you is almost certain
to hate).
\end{verbatim}
\subsection*{Answer}

\begin{verbatim}
top five recommened movies: 
Maya Lin: A Strong Clear Vision (1994) 5.0,
Tough and Deadly (1995) 5.0,
Prefontaine (1997) 5.0,
Aiqing wansui (1994) 5.0,
Saint of Fort Washington, The (1993) 5.0

bottom five recommened movies: 
Girl in the Cadillac (1995) 1.0,
Turbo: A Power Rangers Movie (1997) 1.0,
Vie est belle, La (Life is Rosey) (1987) 1.0,
King of New York (1990) 1.0,
Hostile Intentions (1994) 1.0,
\end{verbatim}
All of the movies recommended here I have not even heard of so I am wondering what is up. 

\section*{4}
\subsection*{Question}
\begin{verbatim}
4.  Choose your (the real you, not the substitute you) favorite and
least favorite film from the data.  For each film, generate a list
of the top 5 most correlated and bottom 5 least correlated films.
Based on your knowledge of the resulting films, do you agree with
the results?  In other words, do you personally like / dislike
the resulting films?
\end{verbatim}


\subsection*{Answer}
I attempted the problem but it did not work well. I should have not tried to do this part in another language.

\section*{4}
Code
\lstinputlisting[language=Scala,frame=single,
caption={Recomender},label=lst:cf,captionpos=b]{Main.scala}  

\end{document}