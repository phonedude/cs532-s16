\documentclass[letterpaper,10pt]{article}
\usepackage[pdftex]{graphicx}
\usepackage{listings}
\usepackage{alltt}
\usepackage{color}
\usepackage{amsmath}
\usepackage{hyperref}
\usepackage{pgfplotstable}
\usepackage{float}


\definecolor{dkgreen}{rgb}{0,0.6,0}
\definecolor{gray}{rgb}{0.5,0.5,0.5}
\definecolor{mauve}{rgb}{0.58,0,0.82}

\hypersetup{
    colorlinks,
    citecolor=black,
    filecolor=black,
    linkcolor=black,
    urlcolor=black
}
\definecolor{lightgray}{rgb}{.9,.9,.9}
\definecolor{darkgray}{rgb}{.4,.4,.4}
\definecolor{purple}{rgb}{0.65, 0.12, 0.82}

\lstdefinelanguage{JavaScript}{
  keywords={typeof, new, true, false, catch, function, return, null, catch, switch, var, if, in, while, do, else, case, break},
  keywordstyle=\color{blue}\bfseries,
  ndkeywords={class, export, boolean, throw, implements, import, this},
  ndkeywordstyle=\color{darkgray}\bfseries,
  identifierstyle=\color{black},
  sensitive=false,
  stringstyle=\color{red}\ttfamily,
  morestring=[b]',
  morestring=[b]"
}

\lstset{
   language=JavaScript,
   backgroundcolor=\color{lightgray},
   extendedchars=true,
   basicstyle=\footnotesize\ttfamily,
   showstringspaces=false,
   showspaces=false,
   numbers=left,
   numberstyle=\footnotesize,
   numbersep=9pt,
   tabsize=2,
   breaklines=true,
   showtabs=false,
   captionpos=b
}

\lstset{
	basicstyle=\footnotesize,
	breaklines=true,
	 backgroundcolor=\color{white},   % choose the background color; you must add \usepackage{color} or \usepackage{xcolor}
  basicstyle=\footnotesize,        % the size of the fonts that are used for the code
  breakatwhitespace=false,         % sets if automatic breaks should only happen at whitespace
  breaklines=true,                 % sets automatic line breaking
  captionpos=b,                    % sets the caption-position to bottom
  commentstyle=\color{dkgreen},    % comment style
  deletekeywords={...},            % if you want to delete keywords from the given language
  escapeinside={\%*}{*)},          % if you want to add LaTeX within your code
  extendedchars=true,              % lets you use non-ASCII characters; for 8-bits encodings only, does not work with UTF-8
  frame=single,	                   % adds a frame around the code
  keepspaces=true,                 % keeps spaces in text, useful for keeping indentation of code (possibly needs columns=flexible)
  keywordstyle=\color{blue},       % keyword style
  otherkeywords={*,grep,sort,head,mv,perl,chmod,...},           % if you want to add more keywords to the set
  numbers=left,                    % where to put the line-numbers; possible values are (none, left, right)
  numbersep=5pt,                   % how far the line-numbers are from the code
  numberstyle=\tiny\color{gray}, % the style that is used for the line-numbers
  rulecolor=\color{black},         % if not set, the frame-color may be changed on line-breaks within not-black text (e.g. comments (green here))
  showspaces=false,                % show spaces everywhere adding particular underscores; it overrides 'showstringspaces'
  showstringspaces=false,          % underline spaces within strings only
  showtabs=false,                  % show tabs within strings adding particular underscores
  stepnumber=1,                    % the step between two line-numbers. If it's 1, each line will be numbered
  stringstyle=\color{mauve},     % string literal style
  tabsize=2,	                   % sets default tabsize to 2 spaces
  title=\lstname                   % show the filename of files included with \lstinputlisting; also try caption instead of title
}


\begin{document} 

\begin{titlepage}

\begin{center}

\Huge{Assignment 6}

\Large{CS532-s16:  Web Sciences}

\Large{Spring 2016}

\Large{John Berlin}

\Large Generated on \today

\end{center}

\end{titlepage}
\newpage
\section*{1}
\subsection*{Question}
\begin{verbatim}

1.  D3 graphing (5 points)

Use D3 to visualize your Twitter followers.  Use my twitter account
("@phonedude_mln") if you do not have >= 50 followers.  For example,
@hvdsomp follows me, as does @mart1nkle1n.  They also follow each
other, so they would both have links to me and links to each other.

To see if two users follow each other, see:
https://dev.twitter.com/rest/reference/get/friendships/show

Attractiveness of the graph counts!  Nodes should be labeled (avatar
images are even better), and edge types (follows, following) should
be marked.

Note: for getting GitHub to serve HTML (and other media types), see:
http://stackoverflow.com/questions/6551446/can-i-run-html-files-directly-from-github-instead-of-just-viewing-their-source

Be sure to include the URI(s) for your D3 graph in your report. 
\end{verbatim}
\newpage
\subsection*{Answer}

The twitter account I choose to visualize was the WebSciDL. Doing so proved to be an undertaking. The python file seen in listing \ref{lst:cf} shows the process. The first thing I do is to get the followers of the WebSciDl group, this was easy as the rate limit for this was not an issue. After getting that comes the hard part.
\newline
\newline
In checking if the followers friends follow each other was not easy. The rate limit for ``reference/get/friendships/show" only let me get through about 3 a day with sleeping to avoid rate limit hits. So I decided to do it a ``longer way". 
So I simply got the friends the normal way and computed the friendship by code. It must be noted that in the method \verb+getFriends+ the json output is simply appended to the file. To correct this you must when finished do so by hand. 
The past assignments we have been asking twitter and I simply re-used the majority of that code to complete this portion.
\newline
\newline
Building the graph was simple enough thanks to using the networkx python library as seen in listing \ref{lst:bwsdlg}.
As usual for more information please see the comments in the listing for details but in short the process is as such.
\begin{enumerate}
\item Get the twitter followers and those followers friends
\item Add the followers as nodes to the directed graph
\item Add WebSciDL edge to all nodes except WebSciDL
\item Add rest of the edges
\item Prepare data for output
\item Write out data
\end{enumerate}
Please not that the image url contains \_normal so I by hand well editor find replace removed it for the full image.
\newline
\newline
The javascript used for the vis is seen in listing \ref{lst:wsdlJS} and the vis can be seen at 
\begin{verbatim}
http://www.cs.odu.edu/~jberlin/WebSciGraphVis/a6.html
\end{verbatim}

\newpage
\section*{2}

\subsection*{Question}
\begin{verbatim}
2.  Gender homophily in your Twitter graph (5 points)

Take the Twitter graph you generated in question #1 and test for
male-female homophily.  For the purposes of this question you can
consider the graph as undirected (i.e., no distinction between
"follows" and "following").  Use the twitter name (not "screen
name"; for example "Michael L. Nelson" and not "@phonedude_mln")
and programatically determine if the user is male or female.  Some
sites that might be useful:

https://genderize.io/
https://pypi.python.org/pypi/gender-detector/0.0.4

Create a table of Twitter users and their likely gender.  List any 
accounts that can't be determined and remove them from the graph.

Perform the homophily test as described in slides 11-15, Week 7.

Does your Twitter graph exhibit gender homophily?
\end{verbatim}

As seen in the table below yes because crossEdge Percent(0.33) is greater than 2pq(0.32).
The code used to generate the results is seen in listing \ref{lst:gnder}
I used a method found on github to ask genderize.io for the results.
The actual data is too large to put in this report and can be found in the file wsdlGenderResults2.csv.
\begin{table}[]
\centering
\caption{Gender Homophilly results}
\label{my-label}
\begin{tabular}{lllllllllll}
males & females & nodes & edges & p    & q    & 2pq  & crossEdges & crossEdgesP & homophily &     \\
120   & 82      & 38    & 120   & 1181 & 0.68 & 0.32 & 0.43       & 386         & 0.33      & Yes
\end{tabular}
\end{table}

\newpage
\newpage
\lstinputlisting[language=Python,frame=single,
caption={Get Twitter Data for the WSDL},label=lst:cf,captionpos=b]{./TwitterFriendShip/checkFriendship.py}      
\newpage
\lstinputlisting[language=Python,frame=single,
caption={Build The WSDL Graph},label=lst:bwsdlg,captionpos=b]{./TwitterFriendShip/buildWSDLGraph.py}
\newpage
\lstinputlisting[language=Python,frame=single,
caption={Check Gender},label=lst:gnder,captionpos=b]{./TwitterFriendShip/genderChecker.py}           
\newpage
\lstinputlisting[language=JavaScript,frame=single,
caption={WebSciDL graph vis},label=lst:wsdlJS,captionpos=b]{./WebSciGraphVis/js/wsdlGraph.js}           
\newpage



\end{document}
