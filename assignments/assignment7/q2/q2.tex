\section{Problem 2}

\subsection{Question}
\vspace*{10pt}
Rank the 1,682 movies according to the 1997/1998 MovieLense
data.  Now rank the same 1,682 movies according to todays (March
2016) IMDB data (break ties based on \# of users, for example: 7.2
with 10,000 raters $>$ 7.2 with 9,000 raters).\\
\\
Draw a graph, where each dot is a film (i.e., 1,682 dots).  The
x-axis is the MovieLense ranking and the y-axis is today's IMDB
ranking.\\
\\
What is Pearon's r for the two lists (along w/ the p-value)?  Assuming
the two user bases are interchangable (which might not be a good
assumption), what does this say about the attitudes about the films
after nearly 20 years?

\subsection{2.1 Answer}
\subsubsection{2.1.1 5-Users Most and Least Correlated to the Substitute Me}
\vspace{3mm}
Question 2 was solved using the code shown in Listing \ref{listing:q2}, with the {\tt sim\_pearson} and the {\tt get\_top} functions from Listings \ref{listing:pearson} and \ref{listing:avgtop}, respectively. The {\tt flatten} function from Listing \ref{listing:flatten} was used to flatted the oddly arranged tuple that was created from the {\tt sim\_pearson} function.

\lstinputlisting[language=Python, caption={Question 2 code}, label=listing:q2, linerange={405-422}, firstnumber={405}]{recommendations.py}

\lstinputlisting[language=Python, caption={sim\_pearson  function}, label=listing:pearson, linerange={45-80}, firstnumber={45}]{recommendations.py}

The results for Question 2 are shown in Table \ref{tab:most} and \ref{tab:least}. 

\begin{table}[h!]
\centering
\begin{tabular}{| l | l |}
\hline
User & Pearson's r \\
\hline
20 & 1.0 \\
74 & 1.0 \\
161 & 1.0 \\
219 & 1.0 \\
284 & 1.0 \\
\hline
\end{tabular}
\caption{Most Correlated Substitute Me}
\label{tab:most}
\end{table}

\begin{table}[h!]
\centering
\begin{tabular}{| l | l |}
\hline
User & Pearson's r \\
\hline
26 & -1.0 \\
39 & -1.0 \\
55 & -1.0 \\
241 & -1.0 \\
300 & -1.0 \\
\hline
\end{tabular}
\caption{Least Correlated Substitute Me}
\label{tab:least}
\end{table}
