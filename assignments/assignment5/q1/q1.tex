\section{Problem 1}

\subsection{Question}
\vspace*{10pt}
\documentclass{article}
\usepackage{hyperref}
\usepackage[procnames]{listings}
\usepackage{color}
\title{Assignment 5}
\date{2016-03-03}
\author{Tarek Fouda}
\usepackage[pdftex]{graphicx}
\usepackage{listings}
\usepackage{alltt}

\definecolor{dkgreen}{rgb}{0,0.6,0}
\definecolor{gray}{rgb}{0.5,0.5,0.5}
\definecolor{mauve}{rgb}{0.58,0,0.82}

\lstset{
	basicstyle=\footnotesize,
	breaklines=true,
}

\begin{document}
  \maketitle


\section{Problem 1} \label{problem1}

\subsection{What is Karate club graph?}

Basically, there was a karate club that consists of 34 individual. The Karate club was observed for a period of three years. At the beginning of the study, the club president, John A., and Mr. Hi argued over the price of the Karate lessons. The president wished to stabiliz the prices while the instructor, Mr. Hi, wished to raise prices. By the time, the whole club split into two parts, one who agrees with the instructor and the other who was pro the president. The Graph shows thefriendship ties between all the members (nodes) until the club was divided into two parts.
\begin{figure}
\centering
\includegraphics[scale=0.95]{1.png}
\caption{The graphic representation of the social relationships among the 34 individuals in the Karate club}
\label{fig:1.png}
\end{figure}
\newpage

Figure \ref{fig:1.png} represents the relationships among the 34 individuals in the Karate club.The edge in this graph represents the interaction or the friendship between each of the nodes.


Figure \ref{fig:2.png} shows the graph in a more convinient way, where the pink nodes are the nodes that support the instructor, who is node 34, on the other hand, the white nodes are the individuals that support node 1, which is the president.



\begin{figure}
\centering
\includegraphics[scale=0.75]{2.png}
\caption{The Karate club}
\label{fig:2.png}
\end{figure}
\newpage
\subsection{Betweeness and grivan newman Algorithm}
The idea of edges betweeness and the Grivan newman algorithm is simple. Initially, if we have a weighted graph we need to calculate the betweeness of all edges and remove the edge with the highest value then recalculate the betweeness  and perform the same steps until the graph is disconnected into two splits.
Keeping in mind that betweeness of the edge is normally the traffic flow in the corresponding edge.
After performing the algorithm on Figure \ref{fig:2.png} and keeping on deleting the nodes with the highest betweeness, we get  Figure \ref{fig:3.png} which proves that it sort of look like the initial Karate club grapghrepresented in Figure \ref{fig:2.png}
\begin{figure}
\centering
\includegraphics[scale=0.75]{3.png}
\caption{The Karate club after spliiting it into two divisions using grivan newman algorithm}
\label{fig:3.png}
\end{figure}

Moreover if we keep on implementing the algorithm on the graph, it will end up splitting into 3 and 4 groups and so on. 
\begin{figure}
\centering
\includegraphics[scale=0.75]{4.png}
\caption{The Karate club after spliiting it into three divisions using grivan newman algorithm}
\label{fig:4.png}
\end{figure}
Figure \ref{fig:4.png} demonstrates that.

\section{References}
The references I used to prove the graphs and to fully understand the concepts were \url{cneurocvs.rmki.kfki.hu/igraph/screenshots2.html} and \url{http://aris.ss.uci.edu/~lin/76.pdf}.
Also grivan newman algorithm was explained well in\url{ http://www-personal.umich.edu/~ladamic/courses/networks/si614w06/ppt/lecture18.ppt}

\url{http://clair.si.umich.edu/si767/papers/Week03/Community/CommunityDetection.pptx}
\end{document}
\clearpage
\subsection{Answer}
As was illustrated in the original study of Zachary's karate club, \cite{wzach77}, a prediction of the structure of the club if a separation were to occur can be made with a high degree of accuracy using weighted edges based on the perceived ``strength'' of each relationship it modeled. The prediction method outlined in the original paper was an implementation of the \textit{maximum flow-minimum cut labeling procedure} \cite{forful62}. The pickled \cite{py:pickle} dataset of the existing karate club, with weights for each edge, was obtained from \url{http://nexus.igraph.org/api/dataset_info?id=1&format=html} and used to create the graph shown in Figure \ref{fig:existing_graph}.

\begin{figure}[h!]
\centering
\fbox{\includegraphics[scale=0.65]{q1/initial_karate_graph.pdf}}
\caption{The Existing Graph}
\label{fig:existing_graph}
\end{figure}

\clearpage

The actual structure of the two resulting clubs after the split are shown in Figure \ref{fig:actual}.

\begin{figure}[h!]
\centering
\fbox{\includegraphics[scale=0.65]{q1/actual_after_split.pdf}}
\caption{Actual Graph After Split}
\label{fig:actual}
\end{figure}

To predict a separation of the existing graph into two or more distinct community components a pair of community detection algorithms will be employed and the results will be compared to the actual results of the split from Zachary's original study \cite{wzach77}. Community Detection was chosen as a means for predicting the results of fission events because it is logical that a given community would less likely be split along strong inter-community edges than those weaker, community-spanning edges.

\clearpage

The first algorithm used was the Edge Betweenness algorithm, developed by Girvan and Newman \cite{girnew04}. This is a divisive algorithm that removes edges that have the highest betweenness measure because these tend to be community-spanning edges. 

\begin{figure}[h!]
\centering
\fbox{\includegraphics[scale=0.5]{q1/clust_eb.pdf}}
\caption{Prediction of Edge Betweenness Algorithm}
\label{fig:graph_eb}
\end{figure}

\begin{verbatim}
Edge Betweenness method results: 
Variant elements:
	[3, 14]
94.12% accuracy
\end{verbatim}

As you can see this method is fairly accurate, with over 94\% of the prediction being correct.
\\
My Girvan-Newman implementation has a $\frac{32}{34} = 94\%$ success rate, making it inferior in this case 
but still effective at predicting almost all of the group memberships. My implementation also predicted that individual $9$ would stay with Mr. Hi, which is the one membership that Zachary missed.

\clearpage

\begin{figure}[h!]
\centering
\fbox{\includegraphics[scale=0.325]{q1/fig1.png}}
\caption{Dendrogram for Girvan-Newman}
\label{fig:output1}
\end{figure}

Table \ref{tab:results} shows the results compared with Zachary's original predictions and the actual data (\emph{Officers'} is John A's faction). Column 5 shows whether my Girvan-Newman implementation resulted in a \emph{Hit} (correctly calculated membership) or \emph{Miss} (incorrectly calculated membership).
\clearpage
\begin{table}
\centering
\small
\begin{tabular}{ | c | p{2cm} | p{2cm} | p{2cm} | p{2cm} | }
\hline
Individual & Actual Group\newline Membership From Split & Zachary's Ford and Fulkerson Procedure Modeled Group\newline Membership From Split & Girvan-Newman Modeled Group\newline Membership From Split & Hit/Miss For Girvan-Newman\\
\hline
1 & Mr. Hi & Mr. Hi & Mr. Hi & Hit \\
\hline
2 & Mr. Hi & Mr. Hi & Mr. Hi & Hit \\
\hline
3 & Mr. Hi & Mr. Hi & Mr. Hi & Hit \\
\hline
4 & Mr. Hi & Mr. Hi & Mr. Hi & Hit \\
\hline
5 & Mr. Hi & Mr. Hi & Mr. Hi & Hit \\
\hline
6 & Mr. Hi & Mr. Hi & Mr. Hi & Hit \\
\hline
7 & Mr. Hi & Mr. Hi & Mr. Hi & Hit \\
\hline
8 & Mr. Hi & Mr. Hi & Mr. Hi & Hit \\
\hline
9 & Mr. Hi & Officers' & Mr. Hi & Hit \\
\hline
10 & Officers' & Officers' & Mr. Hi & Miss \\
\hline
11 & Mr. Hi & Mr. Hi & Mr. Hi & Hit \\
\hline
12 & Mr. Hi & Mr. Hi & Mr. Hi & Hit \\
\hline
13 & Mr. Hi & Mr. Hi & Mr. Hi & Hit \\
\hline
14 & Mr. Hi & Mr. Hi & Mr. Hi & Hit \\
\hline
15 & Officers' & Officers' & Officers' & Hit \\
\hline
16 & Officers' & Officers' & Officers' & Hit \\
\hline
17 & Mr. Hi & Mr. Hi & Mr. Hi & Hit \\
\hline
18 & Mr. Hi & Mr. Hi & Mr. Hi & Hit \\
\hline
19 & Officers' & Officers' & Officers' & Hit \\
\hline
20 & Mr. Hi & Mr. Hi & Mr. Hi & Hit \\
\hline
21 & Officers' & Officers' & Officers' & Hit \\
\hline
22 & Mr. Hi & Mr. Hi & Mr. Hi & Hit \\
\hline
23 & Officers' & Officers' & Officers' & Hit \\
\hline
24 & Officers' & Officers' & Officers' & Hit \\
\hline
25 & Officers' & Officers' & Officers' & Hit \\
\hline
26 & Officers' & Officers' & Officers' & Hit \\
\hline
27 & Officers' & Officers' & Officers' & Hit \\
\hline
28 & Officers' & Officers' & Officers' & Hit \\
\hline
29 & Officers' & Officers' & Officers' & Hit \\
\hline
30 & Officers' & Officers' & Officers' & Hit \\
\hline
31 & Officers' & Officers' & Officers' & Hit \\
\hline
32 & Officers' & Officers' & Mr. Hi & Miss \\
\hline
33 & Officers' & Officers' & Officers' & Hit \\
\hline
34 & Officers' & Officers' & Officers' & Hit \\
\hline
\end{tabular}

\caption{Results of Split, as predicted by my Girvan-Newman Implementation and also compared to Zachary's predictions and the actual data}
\label{tab:results}
\end{table}


\clearpage

The second method used was the Leading Eigenvector algorithm developed by M. Newman \cite{new06}. This method uses a special matrix, called the modularity matrix, to determine which edges to remove.

\begin{figure}[h!]
\centering
\fbox{\includegraphics[scale=0.5]{q1/clust_le.pdf}}
\caption{Prediction of Leading Eigenvector Algorithm}
\label{fig:graph_le}
\end{figure}

\begin{verbatim}
Leading Eigenvector method results: 
Variant elements:
	[]
100.0 % accuracy
\end{verbatim}

This method proves 100\% efficacy in its prediction.

\begin{figure}[h!]
\centering
\fbox{\includegraphics[scale=0.275]{q1/fig2.png}}
\caption{Leading Eigenvector method from Listing \ref{listing:karate}}
\label{fig:output2}
\end{figure}

\clearpage

The python code to produce these graphs is shown in Listing \ref{listing:karate}.

\lstinputlisting[language=Python, caption={Finding Communities in Zachary's Karate Club}, label=listing:karate]{q1/karate.py}