\section{Problem 4}

\subsection{Question}
\vspace*{10pt}
Re-download the 1000 TimeMaps from A2, Q2.  Create a graph where
the x-axis represents the 1000 TimeMaps.  If a TimeMap has \enquote{shrunk},
it will have a negative value below the x-axis corresponding to the
size difference between the two TimeMaps.  If it has stayed the
same, it will have a \enquote{0} value.  If it has grown, the value will be 
positive and correspond to the increase in size between the two
TimeMaps.\\
\\
As always, upload all the TimeMap data.  If the A2 github has the 
original TimeMaps, then you can just point to where they are in 
the report.

\subsection{Answer}
\vspace{2mm}
Using the python script in Listing \ref{listing:gethtml}, 1000 unique URIs were dereferenced and
their raw contents were stored in the {\tt html/raw/} folder as a file with the filename as the
md5-hashed URI. These were then stripped of all html elements and their processed contents were 
stored in the {\tt html/processed/} folder as the same md5-hashed filename. For reference, the URIs were written as the first line of each of their content files.
\vspace{5mm}
\lstinputlisting[language=Python, caption={get\_html.py}, label=listing:gethtml]{q4/get_html.py}
\vspace*{5pt}

\lstinputlisting[language=Python, caption={get\_size.py}, label=listing:getsize]{q4/get_size.py}
\vspace*{5pt}


\textbf{diff} analyzes two files and prints the lines that are different. Essentially, it outputs a set of instructions for how to change one file in order to make it identical to the second file.\\
\\
It does not actually change the files; however, it can optionally generate a script (with the -e option) for the program ed (or ex which can be used to apply the changes.\\
The -e option tells diff to output a script, which can be used by the editing programs ed or ex, that contains a sequence of commands. The commands are a combination of c (change), a (add), and d (delete) which, when executed by the editor, will modify the contents of file1 (the first file specified on the diff command line) so that it matches the contents of file2 (the second file specified).

\lstinputlisting[language=bash, caption={diff command}, label=listing:diff]{q4/diff.sh}


\begin{figure}[h]
\centering
\fbox{\includegraphics[scale=.75]{q4/fig3.png}}
\caption{Change in Size of Processed Files}
\label{fig:processed}
\end{figure}

\begin{figure}[h]
\centering
\fbox{\includegraphics[scale=.65]{q4/fig4.png}}
\caption{Change in Size of Raw Files}
\label{fig:raw}
\end{figure}

