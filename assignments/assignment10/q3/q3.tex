\section{Problem 3}
\subsection{Question}
Re-download the 1000 TimeMaps from A2, Q2.  Create a graph where
the x-axis represents the 1000 TimeMaps.  If a TimeMap has \enquote{shrunk},
it will have a negative value below the x-axis corresponding to the
size difference between the two TimeMaps.  If it has stayed the
same, it will have a \enquote{0} value.  If it has grown, the value will be 
positive and correspond to the increase in size between the two
TimeMaps.\\
\\
As always, upload all the TimeMap data.  If the A2 github has the 
original TimeMaps, then you can just point to where they are in 
the report.

\subsection{Answer}
The python script in Listing \ref{listing:memfind} was used to retrieve the timemaps and then parse the returned html, traveling down the rabbit hole if the target URI has more than 1000 mementos.
\vspace{1mm}
\lstinputlisting[language=Python, caption={mementofinder.py}, label=listing:memfind]{q3/mementofinder.py}

\vspace{5mm}
The dataset created Listing \ref{listing:memfind}. A log scale was used along the y-axis to show more detail among the results. The script in Listing \ref{listing:bld_hist} was used to create the histogram in Figure \ref{fig:hist_ss}, which shows the difference of mementos per site from the dataset.
\vspace{2mm}
\lstinputlisting[language=R, caption={build\_histogram.r}, label=listing:bld_hist]{q3/build_histogram.r}
\vspace{2mm}
\lstinputlisting[caption=Sample of Memento Links, linerange=7-27]
{q3/site_mementos.txt}
\vspace*{5pt}
\begin{figure}[h]
\centering
\fbox{\includegraphics[scale=.85]{q3/histogram.pdf}}
\caption{Histogram of Mementos Count Difference}
\label{fig:hist_ss}
\end{figure}
