\documentclass{article}
\usepackage{hyperref}
\usepackage[procnames]{listings}
\usepackage{color}
\title{Assignment 1}
\date{2016-01-27}
\author{Tarek Fouda}

\begin{document}
  \maketitle
\section{Introduction}
This report mainly discusses my approach and how I implemented and solved each of the three problems assigned to me. I will be discussing every problem in a different section. This assignment is due Thursday 01/28/2016.

\section{Problem 1- cURL}

\subsection{Subsection}

Structuring a document is easy!

\subsubsection{Subsubsection}

More text.

\paragraph{Paragraph}

Some more text.

\subparagraph{Subparagraph}

Even more text.

\section{Problem number 2 - Python}
\paragraph{}
In this problem, it was required to implement a python program that takes a web page such as \href{url}{http://www.cs.odu.edu/~mln/teaching/cs532-s16/test/pdfs.html} and print out all the linksand urls in this web page that are PDFs only!
Iimplemented a python code that takes this webpage from a user and lists all the links that exist in this specific website, but I misunderstood the requirement of this problem and instead of listing all the links that are PDFs, I listed all the URLs and then imported the FPDF library in python to write all the links in a PDF.
\ newpage
\begin{lstlisting}
import urllib2
import BeautifulSoup
import sys
import re
from fpdf import FPDF
print "Please enter the website"
var = raw_input()
print var
#def process(var):
 #  website=urllib2.urlopen(var)
  # html=website.read()
   #links= re.findall("((http|ftp)s>://.*?)",html)
   #print links
def extractingUrls(var):
	
    print "hi"
    if var[0:4]!="http":
        var="http://" + var
    f=(urllib2.urlopen(var)).read()
    k=re.findall('(src|href)="(\S+)"',f)	
    k=set(k)
    print "The Links are:"
    pdf = FPDF()
    pdf.add_page()
    pdf.set_font('Arial', 'B', 10)	
    for x in k:
        if len(x[1])>2:
            print x[1]
            #response = urllib2.urlopen(var)
            #print response.info()
            #print "The size is: ", response.code
            pdf.write(16,x[1]+'and the size of this link equals ' ,'10')
            pdf.write(16,'\n','10')
    pdf.output('tuto1.pdf', 'F')			
extractingUrls(var)
\end{lstlisting}

\end{document}