%%%%%%%%%%%%%%%%%%%%%%%%%%%%%%%%%%%%%%%%%
% Document Author: Plinio H. Vargas
% Course: CS-532, Spring 2016 at Old Dominion University
%
% Structured General Purpose Assignment
% LaTeX Template
%
% This template has been downloaded from:
% http://www.latextemplates.com
%
% Original template author:
% Ted Pavlic (http://www.tedpavlic.com)
%
% Note:
% The \lipsum[#] commands throughout this template generate dummy text
% to fill the template out. These commands should all be removed when 
% writing assignment content.
%
%%%%%%%%%%%%%%%%%%%%%%%%%%%%%%%%%%%%%%%%%
%----------------------------------------------------------------------------------------
%	PACKAGES AND OTHER DOCUMENT CONFIGURATIONS
%----------------------------------------------------------------------------------------

\documentclass{article}

\usepackage{fancyhdr} % Required for custom headers
\usepackage{lastpage} % Required to determine the last page for the footer
\usepackage{extramarks} % Required for headers and footers
\usepackage{floatrow}
\usepackage{graphicx} % Required to insert images
\usepackage{lipsum} % Used for inserting dummy 'Lorem ipsum' text into the template
\usepackage[bookmarks,bookmarksopen,bookmarksdepth=2]{hyperref} % for bookmarks
\usepackage{enumerate}
\usepackage{csquotes} % for quoting things
\usepackage{multirow}
\usepackage{amsmath}
\usepackage{navigator}
\usepackage{caption}
\usepackage[shortlabels]{enumitem}
\usepackage{lmodern}
\usepackage[utf8]{inputenc}
\usepackage[table]{xcolor}% http://ctan.org/pkg/xcolo
\usepackage{longtable}
\usepackage{textcomp}
\usepackage{url}
\usepackage{import}
\usepackage{dashrule} % for dashline
\usepackage{keystroke}

% Margins
\topmargin=-0.45in
\evensidemargin=0in
\oddsidemargin=0in
\textwidth=6.5in
\textheight=9.0in
\headsep=0.25in 

\linespread{1.1} % Line spacing
\newcommand*{\meda}{\mathbin{\scalebox{1.5}{a}}}% increase size of letter a
\newcommand\multibrace[3]{\rdelim\}{#1}{3mm}[\pbox{#2}{#3}]}

% Set up the header and footer
\pagestyle{fancy}
\lhead{\hmwkAuthorName} % Top left header
\chead{\hmwkClass\ (\hmwkClassInstructor\ \hmwkClassTime): \hmwkShortTitle} % Top center header
%\rhead{\firstxmark} % Top right header
\rhead{} % Top right header
\lfoot{\lastxmark} % Bottom left footer
\cfoot{} % Bottom center footer
\rfoot{Page\ \thepage\ of\ \pageref{LastPage}} % Bottom right footer
\renewcommand\headrulewidth{0.4pt} % Size of the header rule
\renewcommand\footrulewidth{0.4pt} % Size of the footer rule

\setlength\parindent{0pt} % Removes all indentation from paragraphs

%----------------------------------------------------------------------------------------
%	DOCUMENT STRUCTURE COMMANDS
%	Skip this unless you know what you're doing
%----------------------------------------------------------------------------------------

% Header and footer for when a page split occurs within a problem environment
\newcommand{\enterProblemHeader}[1]{
\nobreak\extramarks{#1}{#1 continued on next page\ldots}\nobreak
\nobreak\extramarks{#1 (continued)}{#1 continued on next page\ldots}\nobreak
}

% Header and footer for when a page split occurs between problem environments
\newcommand{\exitProblemHeader}[1]{
\nobreak\extramarks{#1 (continued)}{#1 continued on next page\ldots}\nobreak
\nobreak\extramarks{#1}{}\nobreak
}

\setcounter{secnumdepth}{0} % Removes default section numbers
\newcounter{homeworkProblemCounter} % Creates a counter to keep track of the number of problems

\newcommand{\homeworkProblemName}{}
\newenvironment{homeworkProblem}[1][Problem \arabic{homeworkProblemCounter}]{ % Makes a new environment called homeworkProblem which takes 1 argument (custom name) but the default is "Problem #"
\stepcounter{homeworkProblemCounter} % Increase counter for number of problems
\renewcommand{\homeworkProblemName}{#1} % Assign \homeworkProblemName the name of the problem
\section{\homeworkProblemName} % Make a section in the document with the custom problem count
\enterProblemHeader{\homeworkProblemName} % Header and footer within the environment
}{
\exitProblemHeader{\homeworkProblemName} % Header and footer after the environment
}

\newcommand{\problemAnswer}[1]{ % Defines the problem answer command with the content as the only argument
\noindent\framebox[\columnwidth][c]{\begin{minipage}{0.98\columnwidth}#1\end{minipage}} % Makes the box around the problem answer and puts the content inside
}

\newcommand{\homeworkSectionName}{}
\newenvironment{homeworkSection}[1]{ % New environment for sections within homework problems, takes 1 argument - the name of the section
\renewcommand{\homeworkSectionName}{#1} % Assign \homeworkSectionName to the name of the section from the environment argument
\subsection{\homeworkSectionName} % Make a subsection with the custom name of the subsection
\enterProblemHeader{\homeworkProblemName\ [\homeworkSectionName]} % Header and footer within the environment
}{
\enterProblemHeader{\homeworkProblemName} % Header and footer after the environment
}
   
%----------------------------------------------------------------------------------------
%	NAME AND CLASS SECTION
%----------------------------------------------------------------------------------------

\newcommand{\hmwkTitle}{Assignment\ \#1} % Assignment title
\newcommand{\hmwkShortTitle}{Assignment 1} % Assignment title
\newcommand{\hmwkDueDate}{Thursday,\ January\ 28,\ 2016} % Due date
\newcommand{\hmwkClass}{CS-532 Web Science} % Course/class
\newcommand{\hmwkClassTime}{- Spring 2016} % Class/lecture time
\newcommand{\hmwkClassInstructor}{Dr.  Michael L. Nelson} % Teacher/lecturer
\newcommand{\hmwkAuthorName}{Plinio Vargas} % Your name
\newcommand{\hmwkAuthorEmail}{pvargas@cs.odu.edu} % Your name
%----------------------------------------------------------------------------------------
%	TITLE PAGE
%----------------------------------------------------------------------------------------

\title{
\vspace{2in}
\textmd{\textbf{\hmwkClass:\ \hmwkTitle}}\\
\normalsize\vspace{0.1in}\small{Due\ on\ \hmwkDueDate}\\
\vspace{0.1in}\large{\textit{\hmwkClassInstructor\ }}
\vspace{3in}
}

\author{\textbf{\hmwkAuthorName} \\ \hmwkAuthorEmail}
\date{} % Insert date here if you want it to appear below your name

%----------------------------------------------------------------------------------------
%	EMBEDDED FILE
%----------------------------------------------------------------------------------------
\embeddedfile{a1py}{../a1.py}
\embeddedfile{a1html}{../a1.html}
\embeddedfile{a1php}{../a1.php}
%----------------------------------------------------------------------------------------
%	START OF DOCUMENT
%----------------------------------------------------------------------------------------
\begin{document}

\clearpage\maketitle
\thispagestyle{empty}

%----------------------------------------------------------------------------------------
%	TABLE OF CONTENTS
%----------------------------------------------------------------------------------------

%\setcounter{tocdepth}{1} % Uncomment this line if you don't want subsections listed in the ToC

\newpage
\clearpage\tableofcontents
\listoffigures
%\listoftables

\thispagestyle{empty}
\newpage
\setcounter{page}{1}
%----------------------------------------------------------------------------------------
%	Question 1
%----------------------------------------------------------------------------------------
\begin{homeworkProblem} % Custom section title
\vspace*{10pt} % Question
Demonstrate that you know how to use \enquote{curl} well enough to
correctly POST data to a form.  Show that the HTML response that
is returned is \enquote{correct}.  That is, the server should take the
arguments you POSTed and build a response accordingly.  Save the
HTML response to a file and then view that file in a browser and
take a screen shot.\\
\vspace*{5mm}
\\
\centerline{SOLUTION}
http://curl.haxx.se\cite{Manual -- curl usage explained} was an excellent resource to unveil the power of \textbf{\textit{curl}} command. The are many options available with \textit{curl}, including  passing of cookies, which are intensely used by current servers for Cross-Site Request Forgery (CSRF). Then, in order to simplify our demonstration I created a simple html page in my local server that takes two fields: \textit{number1} and \textit{number2}. I named this resource \textit{a1.html}:\\
\\
\import{./}{a1_1.tex}
\newpage
\centerline{\textbf{Screen Shot of html page}}
\begin{center}
	\includegraphics[scale=.6]{images/a1_1.png}
\end{center}
\vspace*{5mm}
Below, is the action script $<$\textit{a1.php}$>$ run by the server when the form is submitted:
\import{./}{a1_1b.tex}
\vspace*{5mm}
\\
If we type from the command line: \textbf{curl -d number1=14 -d number2=11 http://192.168.1.2/a1.php -o a1\_1\_result.html}, the -d option will combine the field names \textit{number1} and \textit{number2} similar to GET method which is in the format]\cite{Manual -- curl usage explained}
$$<variable1>=<data1>\&<variable2>=<data2>\&...$$
The -o option places the response into a given file-name: $<a1\_1\_result.html>$
\newpage
\centerline{\textbf{Screen Shot of curl POST usage}}
\begin{center}
	\includegraphics[scale=.6]{images/a1_1b.png}
\end{center}

\vspace*{10mm}
\centerline{\textbf{Screen Shot of saved HTML response}}
\begin{center}
	\includegraphics[scale=.6]{images/a1_html_result.png}
\end{center}
\end{homeworkProblem}
\newpage
%----------------------------------------------------------------------------------------
%	Question 2
%----------------------------------------------------------------------------------------
\begin{homeworkProblem}
Write a Python program that:
\begin{enumerate}
\item takes as a command line argument a web page
\item extracts all the links from the page
\item lists all the links that result in PDF files, and prints out
     the bytes for each of the links.  (note: be sure to follow
     all the redirects until the link terminates with a "200 OK".)
\item show that the program works on 3 different URIs, one of which
     needs to be: \\
     http://www.cs.odu.edu/~mln/teaching/cs532-s16/test/pdfs.html\\
\\
Python program below (\textbf{\textit{a1.py}}) is attached to this file:\\
\import{./}{a1_2.tex}
\end{enumerate}
\vspace*{5mm}
Below is the result of testing \textit{a1.py} on 3 different URIs: \\
\begin{figure}[!htb]
\caption{Test of \textit{a1.py} at \url{http://www.cs.odu.edu/\string~mln/teaching/cs532-s16/test/pdfs.html} }
\hspace*{-10mm}\includegraphics[scale=.5]{images/cs-532S16_pdf.png}
\end{figure}
\newpage
\begin{figure}[!htb]
\caption{Test of \textit{a1.py} at \url{http://www.cs.odu.edu/} }
\hspace*{-10mm}\includegraphics[scale=.5]{images/cs_odu_pdf.png}
\end{figure}

\vspace*{10mm}
\begin{figure}[!htb]
\caption{Test of \textit{a1.py} at \url{http://www.vbschools.com/curriculum/gifted/} }
\hspace*{-10mm}\includegraphics[scale=.5]{images/vs_schools_gifted_pdf.png}
\end{figure}
\end{homeworkProblem}
\pagebreak
%----------------------------------------------------------------------------------------
%	Question 3
%----------------------------------------------------------------------------------------
\begin{homeworkProblem}
Consider the "bow-tie" graph in the Broder et al. paper (fig 9):\\
http://www9.org/w9cdrom/160/160.html\\
\\
Now consider the following graph:\\
\begin{figure}[!htb]
\CenterFloatBoxes
\begin{floatrow}

\ttabbox
{\begin{tabular}{c c c}
    A &$\to$& B\\
    B & $\to$ & C\\
    C & $\to$ & D\\
    C & $\to$ & A\\
    C & $\to$ & G\\
    E & $\to$ & F\\
    G & $\to$ & C\\
    G & $\to$ & H\\
    I & $\to$ & H\\
    I & $\to$ & J\\
    I & $\to$ & K\\
    J & $\to$ & D\\
    L & $\to$ & D\\
    M & $\to$ & A\\
    M & $\to$ & N\\
    N & $\to$ & D\\
    O & $\to$ & A\\
    P & $\to$ & G
\end{tabular}
}

\ffigbox
{\caption{Bow-Tie Graph Representation} \label{fig:1}}
{\includegraphics[scale=0.25]{images/bow-tie.jpg}}

\end{floatrow}
\end{figure}
\vspace*{5mm}
\\
    For the above graph, give the values for:\\
    
\begin{tabular}{l l}
    \textbf{IN}: \{\textcolor{blue}{M,O,P}\} & \enquote{pages that can reach the SCC, but cannot be reached from it}\cite{Graph structure in the web}\\  
\\    
    \textbf{SCC}: \{\textcolor{blue}{A,B,C,G}\} & \multicolumn{1}{m{12cm}}{\enquote{central core, all of whose pages can reach one another along directed links -- this "giant strongly connected component" (SCC) is at the heart of the web}.}\cite{Graph structure in the web}\\
\\    
    \textbf{OUT}: \{\textcolor{blue}{D,H}\} & \enquote{pages that are accessible from the SCC, but do not link back to it}\cite{Graph structure in the web}\\ 
\\    
    \textbf{Tendrils}: \{\textcolor{blue}{I,J,K,L}\} & \enquote{ pages that cannot reach the SCC, and cannot be reached from the SCC}\cite{Graph structure in the web}\\ 
\\    
    \textbf{Tubes}: \{\textcolor{blue}{N}\} & \enquote{passage from a portion of IN to a portion of OUT without touching SCC}\cite{Graph structure in the web}\\ 
\\    
    \textbf{Disconnected}:\ \{\textcolor{blue}{E,F}\} & Everything NOT fitting all criteria above.\\
\end{tabular}
\end{homeworkProblem}
\begin{thebibliography}{9}
\bibitem{Manual -- curl usage explained} 
Graph structure in the web. (n.d.) Retrieved January 23, 2016, from \url{http://curl.haxx.se/docs/manual.html}
\bibitem{Graph structure in the web}
Graph structure in the web. (n.d.) Retrieved January 23, 2016, from \url{http://http://www9.org/w9cdrom/160/160.html}
\end{thebibliography}
\end{document}